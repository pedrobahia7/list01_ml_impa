\section{Exercício 3b}
\begin{align*}
    Y_i | X_i     & \sim \text{Laplace}(\beta^T x_i, b)                                             \\
    \epsilon      & \sim \text{Laplace}(0, b)                                                       \\
    Y_i | X_i     & = Y = \beta^T X + \epsilon                                                      \\[10pt]
    \mathbb{E}(Y) & = \mathbb{E}(\beta X + \epsilon) = \mathbb{E}(\beta^T X) + \mathbb{E}(\epsilon) \\
                  & = \beta^T X + 0
\end{align*}

\begin{align*}
    \mathbb{V}(Y) & = \mathbb{V}(\beta X + \epsilon) = \mathbb{V}(\beta X) + \mathbb{V}(\epsilon) \\
                  & = 0 + b
\end{align*}

\noindent Toda combinação linear de Laplace são \\
Laplace:

\subsection*{Item b}

\begin{align*}
    \mathcal{L}(\beta) & = - \log ( \ell(\beta|x,y,b) )                                                        \\[10pt]
                       & = - \log \left( \frac{1}{2b} \cdot e^{-\sum_{i}^n \frac{|y_i - x_i\beta|}{b}} \right) \\[10pt]
                       & = - \log(2b) + \sum_{i}^n \frac{|y_i - x_i\beta|}{b}                                  \\[10pt]
                       & = - \log(2b) + \frac{1}{b} \sum_{i}^n |y_i - x_i\beta|
\end{align*}

\begin{itemize}
    \item $-\log(2b)$ é constante e pode ser desconsiderado na otimização
    \item o mesmo pode ser dito sobre o termo que multiplica o somatório
\end{itemize}
\newpage