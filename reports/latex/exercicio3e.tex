\section{Exercício 3e}
\subsection{i}
O modelo com maior passo será o Laplaciano pois o gradiente da função gaussiana
proporcional ao erro, enquanto o gradiente da função Laplaciana não é. Para ela,
O erro define apenas o sinal do gradiente, e não sua magnitude.

\subsection{ii}
Para um mesmo beta e uma amostra de treino, a direção de ambos os gradientes
será a mesma, já que ambos os gradientes dependem apenas do sinal do erro.

Já para novas iterações, a direção pode ser diferente já que ambos os gradientes terão magnitude diferente e resultarão em atualizações diferentes de beta.


\subsection{iii}
Para o caso com mais de uma amostra, a direção do gradiente gaussiano será dada pela soma ponderada de cada erro, enquanto a direção do gradiente Laplaciano
será dada pela soma dos sinais de cada erro.
Assim, a direção do gradiente pode divergir entre os dois modelos, dependendo dos erros de cada amostra.
Por exemplo, se uma amostra tiver um erro muito alto para uma direção enquanto outras duas tiverem erros baixos na direção oposta,
o gradiente gaussiano tenderá a seguir a direção da amostra com maior erro, enquanto o gradiente Laplaciano tenderá a seguir a direção das duas amostras com erros menores.

\newpage