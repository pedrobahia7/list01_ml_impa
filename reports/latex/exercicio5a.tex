\section{Exercício 5b}

\subsection{Métodos de Seleção de Modelos}

Neste exercício, implementamos e comparamos diferentes métodos de seleção de modelos para regressão linear usando o dataset de composição corporal (bodyfat). Os métodos implementados incluem:

\begin{itemize}
    \item \textbf{Best Subset Selection}: Avalia todas as combinações possíveis de features
    \item \textbf{Forward Stepwise Selection}: Adiciona features sequencialmente
    \item \textbf{Backward Stepwise Selection}: Remove features sequencialmente
\end{itemize}

\subsection{Preparação dos Dados}

\begin{lstlisting}[language=Python, basicstyle=\small\ttfamily, breaklines=true]
import statsmodels.api as sm
import numpy as np
import pandas as pd
from sklearn.model_selection import train_test_split, KFold
from sklearn.linear_model import Lasso
from sklearn import preprocessing

# Load and prepare data
bodyfat = pd.read_csv("../data/bodyfat.csv")
X = bodyfat.drop(columns=["BodyFat", "Density"])
y = bodyfat["BodyFat"]

# Split data
X_train, X_test, y_train, y_test = train_test_split(
    X, y, test_size=0.2, random_state=10
)

# Setup cross-validation
kf = KFold(n_splits=5, shuffle=True, random_state=10)
\end{lstlisting}

\subsection{Implementação dos Algoritmos}

Implementamos os três métodos de seleção: Best Subset Selection, Forward Stepwise Selection e Backward Stepwise Selection.

\newpage