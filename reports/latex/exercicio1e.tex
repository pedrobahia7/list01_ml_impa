\section{Exercício 1e}

\textbf{Verdadeira}

A equação dada pela fórmula do intervalo de confiança:

\[ \left[\hat{\beta}_j - 2\sqrt{\hat{\sigma}^2[(X^T X)^{-1}]_{jj}}, \hat{\beta}_j + 2\sqrt{\hat{\sigma}^2[(X^T X)^{-1}]_{jj}}\right] \]

é derivada do fato de $\hat{\beta}$ ter uma distribuição normal. Isso vem do fato da hipótese de que o erro é normal.

Caso ela não seja feita, os intervalos de confiança gerados via \textit{bootstrap} são mais adequados, pois são derivados a partir da distribuição inferida diretamente dos dados. Neste caso, a hipótese não-paramétrica é mais geral e preferível.

\textbf{Justificativa:}

\begin{itemize}
    \item \textbf{Abordagem paramétrica}: Assume que os erros seguem distribuição normal, permitindo o uso da distribuição t de Student para construir intervalos de confiança analíticos.
    \item \textbf{Abordagem não-paramétrica (bootstrap)}: Não assume distribuição específica dos erros, utilizando reamostragem dos dados para estimar a distribuição empírica dos parâmetros.
    \item \textbf{Vantagem do bootstrap}: Mais robusto quando as suposições paramétricas são violadas, especialmente em casos de não-normalidade dos erros.
\end{itemize}

\newpage