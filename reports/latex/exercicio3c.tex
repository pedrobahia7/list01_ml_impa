\section{Exercício 3c}
\noindent $\epsilon \overset{iid}{\sim} \mathcal{N}(0, 1)$ \\
$Y = X\beta + \epsilon$

\begin{align*}
    \ell(y|X\beta, \sigma^2) & = \prod_{i}^n \frac{1}{\sigma\sqrt{2\pi}} \cdot e^{-\frac{1}{2}\left(\frac{y_i - X_i\beta}{\sigma}\right)^2} \\[10pt]
                             & = \frac{1}{\sigma^n (\sqrt{2\pi})^n} \cdot e^{-\frac{1}{2\sigma^2} \sum (y_i - X\beta)^2}
\end{align*}

\begin{align*}
    \mathcal{L}(y|\mu, \sigma^2) & = -\log(\ell(y|\mu, \sigma^2))                                                                                         \\[10pt]
                                 & = -\log \left[ \frac{1}{\sigma^n (\sqrt{2\pi})^n} \cdot e^{-\frac{1}{2\sigma^2} \sum_{i}^n (y_i - X_i\beta)^2} \right] \\[10pt]
                                 & = \log(\sigma^n (\sqrt{2\pi})^n) + \frac{1}{2\sigma^2} \sum_{i}^n (y_i - X_i\beta)^2
\end{align*}

\begin{align*}
    \mathcal{L}(y|\mu, \sigma^2) \approx \sum_{i}^n (y_i - X_i\beta)^2
\end{align*}

\vspace{1em}
\noindent \textbf{Vantagem:} em caso de outliers, erro muito alto, o erro para a amostra $i$ na função ($\ell$) será elevado ao quadrado, resultando em maiores impactos na função de perda. \\
O mesmo não ocorre em b.
\newpage