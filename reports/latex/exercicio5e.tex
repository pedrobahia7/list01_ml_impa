\section{Exercício 5e}

\subsection{Comparação de Erros de Teste}

Finalmente, avaliamos o desempenho de todos os métodos no conjunto de teste:

\begin{lstlisting}[language=Python, basicstyle=\small\ttfamily, breaklines=true]
# Test error evaluation for all methods
test_results = {
    'Ridge': mean_squared_error(ridge_pred, y_test),
    'Backward Selection': mean_squared_error(backward_pred, y_test),
    'Subset Selection': mean_squared_error(subset_pred, y_test),
    'Lasso': mean_squared_error(lasso_pred, y_test)
}

print("Test Error Results:")
for method, error in test_results.items():
    print(f"{method}: {error:.4f}")
\end{lstlisting}

\begin{figure}[H]
    \centering
    \includegraphics[width=0.8\textwidth]{../figures/5/model_selection_test_error_comparison.png}
    \caption{Comparação dos Erros de Teste para Todos os Métodos}
    \label{fig:test_errors}
\end{figure}

\subsection{Ranking dos Métodos}

Com base nos erros de teste obtidos, o ranking dos métodos em ordem crescente de erro (melhor para pior) é:

\begin{enumerate}
    \item \textbf{Backward Selection}: 22.76 - \textcolor{green}{Melhor performance}
    \item \textbf{Subset Selection}: 23.03 - Segundo melhor
    \item \textbf{Ridge Regression}: 23.18 - Terceiro lugar
    \item \textbf{Lasso Regression}: 23.51 - Quarto lugar
\end{enumerate}

\subsection{Análise dos Resultados}

\textbf{Principais observações:}

1. **Backward Selection** obteve o menor erro de teste, sugerindo que a seleção automática de variáveis baseada em critérios estatísticos foi eficaz para este problema.

2. **Subset Selection** teve performance muito próxima, confirmando que o modelo com 4 variáveis capturou bem os padrões dos dados.

3. **Ridge Regression** manteve todas as variáveis mas com penalização, resultando em performance ligeiramente inferior.

4. **Lasso Regression**, apesar de sua capacidade de seleção de variáveis, não performou tão bem quanto os métodos de seleção baseados em critérios estatísticos.

5. A diferença entre o melhor e pior método foi de apenas 0.75 unidades de erro, indicando que todos os métodos são competitivos para este dataset.

\newpage