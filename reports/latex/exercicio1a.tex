\section{Exercício 1a}

\textbf{Falso.} A proximidade entre $\varepsilon_{\text{treino}}$ e $\varepsilon_{\text{teste}}$ pode dar informações sobre o ajuste do modelo aos dados de treino.

Caso $\varepsilon_{\text{treino}}$ seja próximo ao $\varepsilon_{\text{teste}}$, o modelo pode estar \textit{subajustado}, de modo que aumentar a complexidade poderia melhorar sua performance ainda mais.

De maneira análoga, caso o $\varepsilon_{\text{treino}}$ seja menor que o $\varepsilon_{\text{teste}}$, o modelo estará \textit{sobreajustado}, com redução de complexidade podendo resultar em melhoras.

\newpage