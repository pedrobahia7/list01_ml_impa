\section{Exercício 5c}

\subsection{Comparação dos Métodos - R²}

A Figura \ref{fig:model_selection_r2} compara o desempenho dos três métodos em termos de R²:

\begin{figure}[H]
    \centering
    \includegraphics[width=0.85\textwidth]{../figures/5/model_selection_methods_r2_comparison.png}
    \caption{Comparação dos métodos de seleção de modelos - R² vs Número de Features}
    \label{fig:model_selection_r2}
\end{figure}

\textbf{Resultados dos R² por Método:}
\begin{itemize}
    \item \textbf{1 feature}: R² = 0.640 (feature: Abdomen)
    \item \textbf{2 features}: R² = 0.694 (features: Weight, Abdomen)
    \item \textbf{3 features}: R² = 0.710 (adicionando uma terceira feature)
    \item \textbf{Todos os métodos convergem}: Para 1-3 features, todos os métodos encontram as mesmas soluções ótimas
    \item \textbf{Divergência}: A partir de 4+ features, backward stepwise pode encontrar soluções ligeiramente diferentes
\end{itemize}

\newpage